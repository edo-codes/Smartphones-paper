\begin{abstract}

De eerste telefoon die door het grote publiek ‘smartphone’ werd genoemd was de Iphone uit 2007. Sindsdien is er een grote variatie ontstaan, maar smartphones worden allen gekenmerkt door extra functionaliteiten die de telefoon op een mini-computer laten lijken. In de architectuur van de smartphone is gekozen voor processoren en verbindingen die weinig ruimte en energie gebruiken. Dit zorgde voor een lagere complexiteit maar ook minder rekenkracht. De bekenste bestuuringssystemen zijn Android, iOs, Windows Windows Phone/mobile en Symbian. Ze zijn voornamelijk aanpassingen van bestaande besturingssystemen voor computers. Ze verschillen in ontwikkelingstaal en focus op invoer, maar zijn allen gemaakt om met weinig resources zoveel mogelijk te doen. Gebruikersinteractie via de User Interface kenmerkt zich door het veelvuldig gebruik van een touchscreen en versnellingsmeter. Hierdoor zijn, naast toetsenbord en trackpad, ook touch en het draaien van het toestel mogelijkheden voor invoer van smartphones. Door visuele en fysieke feedback is het mogelijk op een klein scherm toch veel functionaliteiten te plaatsen. Omdat veel mensen de smartphone als mini-computer met bebehorende gevoelige informatie gebruiken, is het van belang om data goed te beveiligen. Dit gebeurt via functionaliteiten die in het besturingssysteem zijn ingebouwd maar ook via extra software(functionaliteiten). Deze beveiligingsmethoden richten zich vooral door inbreuk via netwerken. Daarnaast wordt in smartphones gebruik gemaakt van pincodes en beveiligingspatronen om fysieke inbreuk te voorkomen.

\end{abstract}
