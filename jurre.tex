\chapter{User interface}

\section{Evolution}

Om goed te kunnen begrijpen hoe de user interface (UI) van de moderne smartphone tot stand is gekomen moet eerst terug gekeken worden naar de manier waarop de UI eerst opgebouwd is. Voordat het scherm ook nog eens een invoer was werd het user interface zo ontworpen dat deze goed te navigeren was met behulp van de 4 richting toetsen omhoog, links, beneden en recht. Dit betekende dat alle menu's gebaseerd moesten zijn op een directory/lijst type formaat zodat met weinig acties toch veel bestanden en acties bereikt konden worden. Het probleem met zo�n opbouw is dat het heel snel onduidelijk wordt waar alles te vinden is omdat alles verborgen is achter directory's en menu's. Om dit probleem te verhelpen moest een nieuwe manier van invoer komen die navigeren in grotere stappen kan verzorgen. De toevoeging van het Touch screen aan smartphones zorgde ervoor dat de navigatie door de UI op een totaal andere manier mogelijk was. De eerste telefoon die de moderne vorm van UI in smartphones gebruikte was de iPhone. Deze gebruikte het feit dat er met behulp van het Touch screen direct objecten op het scherm geselecteerd konden worden. Hierdoor kan er veel meer op een scherm en dus in een menu vertoond worden zonder dat het een eeuwigheid duurt om iets te selecteren dat helemaal onderin staat. Dit zorgde ervoor dat de UI in veel minder menu�s verdeeld hoefde te worden en er ook meer gefocust kon worden op het uiterlijk van de overgebleven menu�s. Het uiterlijk van de UI gaat sindsdien met sprongen vooruit en ook de functionaliteit wordt steeds beter. \citep{simplicityUI}

\section{Design}

De functie van het user interface is om interactie tussen de gebruiker en het apparaat mogelijk te maken. Het moet er dus voor zorgen dat er mogelijkheden zijn om applicaties te kunnen openen en makkelijk te navigeren naar deze applicaties en instellingen. Ten eerste moet er gekeken worden naar hoe de user naar een applicatie komt en daarna pas naar hoe de applicatie geopend kan worden. In moderne smartphones met een touch screen word er op 2 verschillende manieren door de UI heen genavigeerd. Er kan gescrold worden door een menu waardoor deze meer waarden of objecten kan laten zien. In moderne smartphones wordt dit naast de bekende manier van naar beneden gaan in een lijst op nog een andere manier gedaan die beter bekend is als swipen. Dit swipen laat meestal de ``volgende pagina'' van het menu zien door naar links of naar rechts te slepen op het scherm en is eigenlijk gewoon een variant van scrollen. Naast deze manier van navigeren die binnen menu�s gebeurt moet er ook een manier zijn om van menu naar menu te gaan. Dit gaat meestal op dezelfde manier waarop applicaties geselecteerd worden en verschilt weinig van de oude manier in de UI. Het selecteren van objecten of starten applicaties gebeurt simpelweg door op de grafische representatie te drukken. \citep{guidelinesUI}  Nadat deze functies mogelijk zijn kan gekeken worden naar de manier waarop alles mooi en duidelijk op het scherm te zien is. Het uiterlijk van een user interface is even belangrijk als de mogelijke functionaliteit die deze levert. Het is namelijk erg moeilijk om functies te gebruiken als deze onduidelijk staan op het scherm. Voor navigatie is het erg belangrijk dat alle objecten die op het scherm staan duidelijk van elkaar te onderscheiden zijn. In de moderne smartphone UI wordt dit gedaan door de objecten weer te geven met iconen en daaronder een benaming. Daarnaast moeten de overgangen van menu naar menu of applicatie mooi zijn. \citep{designUI} 

\section{Invoer}

Het touch screen levert een gigantisch aantal mogelijkheden voor de gebruiker om het user interface te gebruiken. Hierdoor kan deze manier van invoer gezien worden als de belangrijkste manier van invoer in moderne smartphones. Het touch screen wordt gebruikt om te navigeren in menu's, om objecten te selecteren, applicaties te starten en nog veel meer. Navigeren wordt voornamelijk gedaan door het bewegen op het scherm van boven naar beneden of andersom (scrollen) en door bewegen van links naar recht of andersom (swipen). Deze zijn beide invoermogelijkheden met maar een enkele vinger, terwijl het scherm meerdere vingers aan kan. Het gebruiken van twee punten levert de mogelijkheid om veel gebruikte functies zoals inzoomen en uitzoomen mogelijk te maken met een enkele actie. Deze mogelijkheden zorgen ervoor dat het gebruik  van het apparaat veel makkelijker gaat. De volgende invoer mogelijkheid die een touch screen levert is het gebruik van een virtueel toetsenbord. Deze mogelijkheid is erg belangrijk aangezien de meeste smartphones geen individueel toetsenbord meer hebben. Zodra de user interface detecteert dat toetsenbord invoer nodig is roept deze een simpel virtueel toetsenbord. Om het toetsenbord nog te kunnen gebruiken kunnen er maar weinig toetsen op het scherm vertoond worden en zijn er dus knoppen om een andere layout te laden.  Naast het touch screen is er nog een vaak over het hoofd geziene invoer mogelijkheid. Als het toestel gedraaid wordt draait het scherm op zo�n manier dat deze recht op staat.  De manier waarop de ori�ntatie van het apparaat bepaald wordt is met behulp van een accelerometer, deze detecteert ook beweging en is daarmee dus ook een invoer mogelijkheid. Er kan bijvoorbeeld met behulp van een snelle beweging van het apparaat naar rechts terug gegaan worden naar het beginscherm . Er zijn zelfs nog meer mogelijkheden door het combineren van het touch screen en de beweging.  Nadat een actie uitgevoerd is moet er feedback zijn over wat deze actie betekend. \citep{designUI}

\section{Feedback}

Er zijn meerde types feedback die geleverd kunnen worden aan de gebruiker om duidelijk te maken dat een actie uitgevoerd is. De belangrijkste hiervan is de visuele feedback die op het scherm te zien is. Het mooie aan visuele feedback is dat het direct duidelijk is op het scherm wat voor een actie er uitgevoerd gaat worden, zelfs als het nog niet uitgevoerd is. Bijvoorbeeld als in de apple iPhone UI een icoon aangeklikt wordt opent de UI deze link. Maar als het icoon langdurig aangeklikt wordt zonder los te laten gaat het icoon langzaam schudden om aan te geven dat de actie nu een ander effect heeft, in dit geval kan het icoon daarna verschoven/verplaatst worden. Deze simpele visuele feedback maakt het duidelijk dat dezelfde invoer een andere actie uitvoert.  Naast visuele feedback is een andere belangrijke feedback de fysieke feedback. Deze wordt voornamelijk gebruikt als een manier van directe feedback bij het gebruik van een virtueel toetsenbord. Als een toets ingedrukt wordt vibreert het apparaat om deze actie duidelijk te maken. Het is belangrijk om feedback te krijgen om fouten te ontdekken en voorkomen. 


