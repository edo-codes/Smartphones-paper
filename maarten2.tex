\chapter{Conclusie}

Het begin van de mobiele telefonie is te herleiden op de tweede wereldoorlog. Rond 1990 werd de mobiele telefoon steeds vanzelfsprekender en rond die tijd is ook de eerste vorm van een smartphone op de markt gebracht. De eerste telefoon die onder het grote publiek bekendheid kreeg als `smartphone' was de eerste iPhone uit 2007. Sindsdien is de telefoon steeds verder ontwikkeld tot de `smartphone' die we nu kennen en de features die we erbij verwachten.

De architectuur van de smarthphone lijkt in het algemeen erg op die van een computer. Door andere eisen aan de smartphone, zoals klein formaat, beperkte grootte van het scherm en beperkte batterijcapaciteit, zijn er echter wel consessies gedaan aan functionaliteiten in de smartphone. Zo is de keuze gemaakt voor Reduced Instruction Set Computer (RISC) processoren, die minder energie gebruiken en minder hitte produceren. Daarnaast is de architectuur van het systeem een stuk kleiner gemaakt door gebruik van System on a chip (SoC) –architecturen. Deze combineren CPU, GPU, bussen en geheugencontroller in één chip. Ook andere hardware kan geïntegreerd worden, wat zorgt voor een reductie van complexiteit en ruimtegebruik. In het uitvoeren van software is met dezelfde eisen rekening gehouden. Zo mag er, enkele uitzonderingen daargelaten, maximaal \'e\'en app tegelijk draaien en beslaat de UI van een app het gehele scherm. Vergeleken met PC's is de manier waarom software (apps) gepubliceerd wordt erg beperkt. In de meeste besturingssystemen heeft de fabrikant controle over welke apps gepubliceerd worden. Dit heeft voordelen voor de veiligheid van het systeem en de kwaliteit van het softwareaanbod maar beperkt de functionaliteit die apps kunnen bieden.

Ook de besturingssystemen zijn afgeleid van normale computers, maar speciaal voor smartphones aangepast. Op dit moment is de Android het meest voorkomende besturingssysteem voor smartphones. Android maakt gebruik van een Linux kernel en applicaties in Java. Android geeft een hoge mate in vrijheid als het gaat om aanpassingsvermogen van het bestuuringssysteem. Dit in tegenstelling tot iOS, gebaseerd op Mac OS X, het besturingssysteem voor de iPhone. Applicaties in iOS hebben strenge regels waar ze aan moeten voldoen en applicaties kunnen alleen vanuit de App store geïnstalleerd worden. De iOS is speciaal ontwikkeld om alleen te besturen door middel van touch, wat ook terug te zien is in het uiterlijk van iOS smartphones. Blackberry OS, ontwikkeld door RIM, wordt gebruikt op Blackberry telefoons. De meeste blackberry's hebben een volledig toetsenbord en een trackpad, waar de OS op gebaseerd is. Veel BlackBerry's zijn voor zakelijk gebruik, en dat is ook te zien aan de extra functionaliteiten die het OS biedt, zoals mailservice en versleuteld dataverkeer. Applicaties op de Blackberry zijn, net als bij de Android, gebaseerd op Java. Een andere, opkomende, OS voor smartphones is de Windows Phone/Mobile. De eerste versies van dit OS waren gebaseerd op Windows CE, een aangepaste versie speciaal voor mobiele apparaten. Sinds Windows Phone 8 is dit OS gebaseerd op Windows NT, waar ook computerbesturingssystemen op gebaseerd zijn. Windows Phone/Mobile wordt voornamelijk bestuurd door touch en maakt gebruik van grote aanraakvlakken. Een andere, minder bekende, OS is Symbian, een besturingsysteem dat vooral door Nokia werd gebruikt. Tegenwoordig gebruikt Nokia Windows Phone. Symbian gebruikt vooral besturing via een trackpad maar heeft ook touch functies. Sybian en bijbehordende applicaties zijn in C++ geschreven.

Naast de bestuuringssyetemen is er in de User Interface ook rekening gehouden met de andere eisen die gesteld worden aan een smartphone. Vanuit de normale mobiel is de User Interface altijd bestuurd met 4 richtingstoetsen. Tegenwoordig gebruiken bijna alle smartphones touchscreen, waarbij in minder handelingen meer gegevens bereikt kunnen worden. Het touch screen kan van boven naar beneden genavigeerd worden (scrollen) en van links naar rechts (swipen). Andere invoermogelijkheden zijn een toetsenbord, trackpad of het draaien van het toestel. Opstarten van applicaties gaat snel door op de bijbehorende grafische representatie te drukken. Door visuele feedback te geven aan de gebruiker zijn op een beperkt scherm meerdere functies mogelijk, zoals een trillend icoontje dat aangeeft dat het verplaatst kan worden. Ook door vibratie van het toestel kan er feedback aan de gebruiker gegeven worden. Voor navigatie is het user interface erg belangrijk, omdat het duidelijk moet zijn welke functie zich waar bevindt, hier wordt bij smartphones dan ook veel aandacht aan besteed. 

Omdat men de smartphone tegenwoordig vaak gebruikt als minicomputer is privacy en beveiliging ook erg van belang. Beveiling van een smartphone bestaat uit het beveiligen van de data op de smartphone en het beveiligen van de dataoverdracht. Beveiliging zit vaak al in het OS ingebouwd, zoals proces isolatie, filesystem permissions en geheugenbescherming. Daarnaast zijn er hogere softwarelagen die gebruikt worden voor beveiliging, zoals antivirus programma's, firewalls, visuele notificaties en de Turing test, een test om te kijken of de gebruiker een machine of een mens is (bijvoorbeeld CAPTCHA). Deze softwarelagen beschermen tegen kwaadwillenden door aanvallen uit externe bronnen tegen te houden, zoals WiFi, MMS, de browser en applicaties. Bescherming tegen andere gebruikers die fysiek data van de telefoon willen halen, gebeurd via pincodes op simkaart en telefoon. Tegenwoordig wordt als schermbeveiliging van de telefoon vaak een patroon gekozen die de eigenaar op de telefoon moet tekenen, omdat dit makkelijker te onthouden is voor mensen.
