\chapter{Besturingssystemen}

Voor computers is het besturingssysteem een essentieel onderdeel voor een werkend systeem. Zo ook bij smartphones, smartphones zijn in de basis een kleine pc dus is het logisch dat ze ook een besturingssysteem nodig hebben om te kunnen werken.  Smartphones bestaan nog niet zo lang en daarom is er nog veel concurrentie tussen verschillende producenten van smartphone besturingssystemen.

De besturingssystemen zijn:

\section{Android}

Android is samen met iOS het grootste besturingssysteem voor een smartphone. De grootste aandeelhouder van Android is Google, nadat Google het had overgenomen van Android inc. . Het besturingssysteem is gebaseerd op de kernel van Linux en de applicaties die voor Android gemaakt zijn zijn gemaakt met Java. Het besturingssysteem is grotendeels bedoeld om te bedienen door middel van touchscreen, maar ook met een toetsenbord en trackpad kan Android bestuurd worden. Android is gemaakt om gebruikt te worden op smartphones, maar in de loop van de tijd zijn daarbij ook tablets bij gekomen en er zijn ook computers waarop Android geïnstalleerd staat. Het besturingssysteem was eerst alleen ontwikkeld om op ARM-instructieset te werken, maar sinds Android 4.0 is Android gaan samenwerken met Intel om er voor te zorgen dat Android ook kan werken op het X86-instructieset. Android staat er om bekend dat er veel aangepast kan worden aan het besturingssysteem, dat kan omdat het besturingssysteem opensource is en het besturingssyteem laat veel toe. Fabrikanten van smartphones passen Android aan om verschillend te zijn van de andere fabrikanten en soms ook om Android te verbeteren. Er zijn ook verschillende community's die Android aan passen en die stellen ze dan beschikbaar aan het publiek, ze zorgen er ook voor dat de meeste smartphones geupgrade kan worden naar een nieuwere versie, want de producenten van smartphones geven alleen hun topmodellen vaak updates. Het is bij Android ook mogelijk om applicaties buiten de play store om te installeren. Dat heeft tot gevolgen dat applicaties ook van het internet te halen zijn. Je kunt daardoor als developer van een applicatie er voor kiezen om je applicatie via andere wegen aan te bieden. Aan de andere kant zorgt het ook voor meer piraterij omdat de applicaties door onbevoegde gratis aangeboden kunnen worden. De mogelijkheid dat je buiten Android om applicaties te installeren word ook gebruikt om malware op Android telefoons te installeren zonder dat ze het door hebben. De Google play store(Android market) werd eerst niet door Google gecontroleerd op malware, maar dit hebben ze later wel toegevoegd.


\section{iOS}

iOS is het besturingssysteem die Apple gebruikt voor zijn smartphones, ze zijn ook de enige die het gebruikt. iOS is het eerste besturingssysteem voor smartphones die alleen op gericht is om te besturen door middel van touch, dat zie je ook terug in het uiterlijk van iOS, want je hebt verschillende schermen waar je tussen kan switchen en die kan je vullen met icoontjes waarmee je applicaties kan opstarten. In die tijd was dat revolutionaire en iOS is daardoor nu ook zeer populaire. iOS is gebaseerd op Mac osx, dat is het besturingssysteem van Apple voor hun computers. De applicaties voor iOS worden geschreven in objectiv c. iOS was eerst aangekondigd voor de iPhone, de i in iOS staat ook voor iPhone, en later ook nog voor de IPodtouch, de IPad en ook de Apple tv. iOS is een gesloten systeem, je hebt weinig mogelijkheden om het besturingssysteem aan te passen en de applicaties die voor iOS ontwikkeld worden moeten aan bepaalde regels van Apple voldoen om in de app store te komen. Het is wel mogelijk om via andere manieren applicaties te installeren, maar dan moet je door middel van beveiligings problemen in iOS het besturingssysteem aan passen waardoor je extra mogelijkheden kan krijgen. In de app store van iOS zijn momenteel de meeste applicaties beschikbaar voor mobiele besturingssystemen. 

\section{Blackberry OS}

Blackberry OS is ontwikkeld door RIM en word ook door RIM gebruikt voor hun smartphones, de Blackberry. Voor het besturen van het besturingssysteem gebruikten ze eerst de trackwheel, daarna zijn ze gebruik van de trackpad gaan maken, ze gebruiken dat nu nog steeds maar nu is het op sommige smartphones van BlackBerry ook mogelijk om het te bedienen via het touchscreen. BlackBerry OS is vooral gericht op de zakelijke markt en daarom hebben ook de meeste BlackBerry onder het scherm een volledig toetsenbord zitten, zo kunnen zakelijke mensen die onderweg zijn snel een email typen. RIM heeft ook enterprise service voor zakelijk gebruik ontwikkelt, ze hebben dat ontwikkelt zodat gebruikers hun bedrijfsmail kunnen koppelen aan hun Blackberry. Ze hebben ook internet service zodat je gebruik kan maken van hun internet browser en hun mail applicatie, daarmee kan je ook mailadressen mee beheren zonder dat jezelf een mailserver nodig hebt, als je een van de twee abonnementen hebt dan gaat het verkeer ook via de servers van RIM en dat is zakelijk interessant omdat het verkeer tussen de server en de Blackberry versleuteld is. Het is niet verplicht om op een van de twee diensten van Blackberry te abonneren, maar dan kan de gebruiker geen gebruik maken van de applicaties van Blackberry die verbinding maken met internet. Blackberry was de afgelopen jaren zeer populaire onder niet zakelijke gebruikers, omdat je met de Blackberry internet service de applicatie ping kon gebruiken, maar sinds de komst van whatsapp, wat goedkoper is dan de dienst van Blackberry, is de populariteit van Blackberry afgenomen. Het Blackberry OS is gebaseerd op Java. In het begin had RIM geen winkel waar je de applicaties kon downloaden, maar ze boden wel de mogelijkheid om third party applicaties te downloaden en te instaleren.

\section{Windows Mobile en Windows Phone}

Windows Phone/mobile is ontwikkelt door Microsoft. Windows Phone is de opvolger van Windows mobile, maar ze zijn allebei gebaseerd op een versie van Windows CE, dat is een aangepaste versie speciaal van Windows voor mobiele apparaten. Met de komst van Windows Phone 8 is dat alleen anders, want die is gebaseerd op Windows NT, daar zijn ook de Windows versies voor de computer op gebaseerd. Windows phone word bestuurd door middel van touch en heeft net als de xbox360 en windows8 een interface met veel grote vlakken, Microsoft noemt het de metro interface. Microsoft maakt zelf geen smartphones voor het besturingssysteem, maar ze laten smartphone producten smartphones produceren. Microsoft stelt wel eisen aan de resolutie van het scherm en de hardware die de producenten gebruiken. De producenten proberen zich te onderscheiden door verschillende scherm grootte toe te passen, dezelfde processor op verschillende snelheid te clocken en verschillende designs toe te passen. Toen Microsoft over ging van Windows mobile naar Windows Phone waren de applicaties niet compatibel met elkaar, van Windows phone 7 naar 8 is dat wel het geval, maar de smartphones die op Windows Phone 7 draaien krijgen geen update naar Windows Phone 8.
 
 
\section{Symbian}

Symbian is het besturingssysteem dat opgericht is door symbian foundation,  veel groten bedrijven doen mee met de foundation. De belangrijkste deelnemer van de foundation is Nokia, Nokia heeft ook de foundation overgenomen in 2010. Symbian werd door veel producenten van smartphones gebruikt, maar sinds de komst van Android zijn er veel smartphone producenten overgestapt naar Android. Nokia was de laatste grote producent die symbian gebruikte, maar die zijn volledig overgestapt op Windows phone. Symbian is te besturen via trackpad, maar de het is ook mogelijk om het via touch te bieden. Symbian is gemaakt in C++ en de applicaties die je voor symbian kan maken zijn ook in C++ geschreven.
